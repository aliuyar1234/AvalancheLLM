
\documentclass[10pt]{article}

% --- Formatting (ICLR-like two-column layout without relying on external style files) ---
\usepackage[letterpaper,margin=0.78in,columnsep=0.25in]{geometry}
\setlength{\parindent}{0pt}
\setlength{\parskip}{4pt}
\setlength{\emergencystretch}{1.5em}

\usepackage{times}
\usepackage{microtype}

\usepackage{amsmath,amssymb,amsfonts}
\usepackage{booktabs}
\usepackage{siunitx}

\usepackage{graphicx}
\usepackage{subcaption}
\usepackage{float}

\usepackage{longtable}
\usepackage{array}

\usepackage[numbers,sort&compress]{natbib}
\usepackage[colorlinks=true,allcolors=blue]{hyperref}
\usepackage[nameinlink,noabbrev]{cleveref}
\usepackage{xurl}

\usepackage{caption}
\captionsetup{font=small,labelfont=bf}

\usepackage{enumitem}
\setlist[itemize]{leftmargin=*,topsep=2pt,itemsep=2pt}
\setlist[enumerate]{leftmargin=*,topsep=2pt,itemsep=2pt}

\title{\vspace{-0.2in}Token--Layer Activation Event Cascades in LLMs: Rate-Matched Connectivity under Gain Scaling}
\author{Ali Uyar\\Independent Researcher}
\date{}

\begin{document}

\twocolumn[
\maketitle
\vspace{-0.15in}
\begin{abstract}
Activation patterns in large language models are often studied via static sparsity or outlier magnitudes, but less is known about how \emph{activation events} form connected cascades across token positions and depth. We model gated-MLP activations as a token-by-layer \emph{event lattice} by standardizing a fixed internal tensor and thresholding it into sparse binary events. This enables connected-component analysis that yields avalanche-like cascades on a two-dimensional lattice. We introduce directional branching metrics that decompose local propagation into token-direction and depth-direction components, and we evaluate a low-compute gain intervention that scales each layer's MLP residual contribution at inference time.

To isolate connectivity effects from trivial rate effects, we design two strong controls: (i) \emph{per-layer rate-matched thresholds} that equalize marginal event rates across gains, and (ii) a \emph{marginals-preserving raster shuffle} that permutes token order within each layer while preserving per-layer event counts exactly. We avoid equating heavy-tailed component statistics with criticality by treating tail fits as descriptive diagnostics and using multiple falsifiable signatures. Finally, we calibrate a gain $g^{\star}$ on Dataset~A using a mechanistic criterion based on branching, then evaluate the same $g^{\star}$ unchanged on Dataset~B and ARC multiple-choice, comparing against the baseline $g{=}1$.

In the provided 7B runs, total branching $b_{\mathrm{tot}}$ varies with gain under rate matching (Table~\ref{tab:t01_selected} and \Cref{fig:F03_BRANCHING_CURVES}) and the null-controlled residual $\Delta b_{\mathrm{tot}}$ remains non-zero (\Cref{fig:F04_NULL_DELTAB}). This framework supports both positive and negative transfer outcomes under strong controls by reporting mechanistic signatures alongside task metrics.
\end{abstract}
\vspace{0.1in}
]

\section{Introduction}
Large language models (LLMs) exhibit rich internal activation structure, including sparsity patterns, context-dependent gating, and occasional extreme outliers. However, many analyses treat activations as independent samples or focus on layerwise distributions, which can obscure \emph{connectivity in token-by-depth space}. Motivated by cascade analyses in neuroscience \citep{beggs2003neuronal}, we ask a simple mechanistic question: \emph{when we threshold a fixed internal tensor into sparse activation events, do those events form connected cascades across token position and layer depth, and how does that connectivity change under controlled perturbations?}

We propose a concrete measurement construct: a token-by-layer event raster derived from standardized gated-MLP activations. This event lattice supports connected-component analysis (``avalanches'') and directional branching metrics that quantify local propagation along token and depth directions. We then study a global \emph{gain} intervention that scales the MLP residual contribution by a scalar $g$ at inference time. Because $g$ trivially rescales activations and thus event \emph{rates}, we include reviewer-proof controls that (i) match marginal event rates per layer across $g$ and (ii) destroy within-layer temporal structure while preserving marginals exactly. We use these controls to probe quasi-critical-like regimes without claiming a phase transition or brain equivalence.

\textbf{Contributions.}
\begin{itemize}
  \item \textbf{Event lattice construct.} A token-by-layer binary event raster derived from a fixed standardized gated-MLP tensor, enabling avalanche-style connected-component analysis in transformers.
  \item \textbf{Directional branching decomposition.} Metrics $b_{\text{time}}$, $b_{\text{depth}}$, and $b_{\text{tot}}$ that quantify local propagation along token and depth directions, plus a marginals-controlled residual $\Delta b$.
  \item \textbf{Reviewer-proof controls.} Per-layer rate-matched thresholds $\tau_{\ell}(g)$ and a post-hoc within-layer time-permutation null that preserves marginals exactly.
  \item \textbf{Mechanistic calibration and transfer test.} Calibrate $g^{\star}$ on Dataset~A by minimizing $|b_{\mathrm{tot}}(g)-1|$ over a fixed gain grid (allowing boundary solutions), then evaluate unchanged on Dataset~B and ARC multiple-choice against $g{=}1$.
  \item \textbf{Conservative quasi-critical probing.} Multiple signatures used as falsifiable probes; heavy-tail fits are reported as descriptive only \citep{clauset2009power,touboul2010can}.
\end{itemize}

\section{Related work}
We draw inspiration from neuronal avalanche analyses and branching measures \citep{beggs2003neuronal} while avoiding any claim of brain equivalence. We also follow statistical cautions that power-law-like tails alone are insufficient evidence of criticality \citep{clauset2009power,touboul2010can}. In deep learning, ``edge-of-chaos'' style analyses study signal propagation and criticality-like regimes in random networks and deep models \citep{schoenholz2016deep,poole2016exponential,pennington2017resurrecting}; our work instead operates at inference time on trained transformers and uses rate-matched binary event rasters. Finally, we focus on gated MLP activations (common in modern LLMs; \citep{shazeer2020glu}) and distinguish event connectivity from activation-magnitude outlier studies \citep{sun2024massive}.

\section{Method}
\subsection{Token-by-layer event lattice}
Let a transformer with $L$ blocks process a token sequence of length $T$. For each token position $t\in\{1,\dots,T\}$ and layer $\ell\in\{1,\dots,L\}$, we extract a fixed internal gated-MLP tensor $u_{t,\ell,i}$ (indexed by MLP hidden dimension $i$) at a specified hookpoint (pre-down-projection for gated MLPs). We standardize using per-layer moments $(\mu_{\ell},\sigma_{\ell})$ estimated on a fixed calibration slice:
\[
z_{t,\ell,i} = \frac{u_{t,\ell,i}-\mu_{\ell}}{\sigma_{\ell}+\varepsilon}.
\]
We define spike events in two ways:
\[
s^{(+)}_{t,\ell,i}=\mathbb{1}[z_{t,\ell,i}>\tau_{\ell}] \quad\text{and}\quad
s^{(\pm)}_{t,\ell,i}=\mathbb{1}[|z_{t,\ell,i}|>\tau_{\ell}].
\]
We aggregate spikes into an event-count field and a binary occupancy field:
\[
A_{t,\ell} = \sum_{i} s_{t,\ell,i}, \qquad X_{t,\ell} = \mathbb{1}[A_{t,\ell} > 0].
\]
The binary raster $X\in\{0,1\}^{T\times L}$ is the event lattice used for connected components.

\subsection{Rate-matched thresholds}
For each gain $g$ and layer $\ell$, we choose $\tau_{\ell}(g)$ so that the marginal event rate matches a target $r^{\star}$:
\[
\mathbb{E}_{t,i}\left[s_{t,\ell,i}\right] \approx r^{\star}.
\]
Operationally we compute $\tau_{\ell}(g)$ as a per-layer quantile of $z_{t,\ell,i}$ on the calibration slice (separately for $s^{(+)}$ and $s^{(\pm)}$). Rate-matching success is verified by the maximum absolute rate error across layers (\Cref{fig:F02_RATE_MATCH_CHECK}).

\subsection{Avalanches and connected components}
We define avalanches as connected components of active sites in the token-by-layer lattice under a 4-neighborhood adjacency (time and depth moves). Each component $C$ has size $S(C)=\sum_{(t,\ell)\in C} A_{t,\ell}$, duration in tokens (span along $t$), and depth span along $\ell$.

\subsection{Directional branching metrics}
For each active site $(t,\ell)$ we count forward-neighbor activations in time $(t+1,\ell)$ and depth $(t,\ell+1)$, normalized by the number of possible forward neighbors. Aggregating yields:
\[
b_{\text{time}},\quad b_{\text{depth}},\quad b_{\text{tot}}=b_{\text{time}}+b_{\text{depth}}.
\]
We also compute a susceptibility proxy $\chi$ (variance-based; see Appendix) and a descriptive crackling exponent fit on the avalanche size distribution (reported with bootstrap confidence intervals). We treat these as \emph{signatures} rather than proofs of criticality.

\subsection{Marginals-preserving null and \texorpdfstring{$\Delta b$}{Delta b}}
To isolate connectivity from marginals, we construct a within-layer time-permutation null: for each layer $\ell$, apply a permutation $\pi_{\ell}$ to token indices, permuting $A_{t,\ell}$ across $t$ while preserving each layer's multiset of counts exactly. We compute branching on this permuted raster to obtain $b_{\cdot,\text{perm}}$, then define:
\[
\Delta b_{\cdot} = b_{\cdot} - b_{\cdot,\text{perm}}.
\]
A non-zero $\Delta b$ indicates structure beyond marginals.

\subsection{Gain intervention and mechanistic \texorpdfstring{$g^{\star}$}{g*}}
We modify each transformer block's residual update to scale the MLP branch:
\[
h_{\ell+1} = h_{\ell} + \mathrm{Attn}_{\ell}(h_{\ell}) + g\cdot \mathrm{MLP}_{\ell}(h_{\ell}).
\]
On Dataset~A, for each (spike definition, target rate) condition, we select
\[
g^{\star} = \arg\min_{g\in \mathcal{G}} |b_{\mathrm{tot}}(g) - 1|.
\]
We then evaluate the same $g^{\star}$ unchanged on Dataset~B and ARC multiple-choice, comparing against $g{=}1$.

\section{Experiments}
\subsection{Model and datasets}
We evaluate two 7B checkpoints from the same model family (Qwen2.5-7B-Instruct and Qwen2.5-7B base) using three datasets:
\begin{itemize}
  \item \textbf{Dataset A:} a fixed slice of Wikitext-103 validation (mechanistic calibration and signatures).
  \item \textbf{Dataset B:} a fixed slice of C4-en validation (transfer evaluation).
  \item \textbf{ARC multiple-choice:} ARC-Challenge (task metric evaluation).
\end{itemize}
All experiments are inference/analysis heavy and fit within a single-GPU budget by limiting the number of sequences and gain conditions in the published run.

\subsection{Conditions and controls}
We evaluate:
\begin{itemize}
  \item Two spike definitions: one-sided $s^{(+)}$ and two-sided $s^{(\pm)}$.
  \item Target marginal rates: $r^{\star}\in\{1,2,4,8\}\times 10^{-5}$.
  \item Gain grid $\mathcal{G}=\{0.70,0.80,0.85,0.90,\allowbreak 0.95,1.00,1.05,\allowbreak 1.10,1.15,1.20,1.30\}$.
\end{itemize}
For each $g$ we rate-match $\tau_{\ell}(g)$ and evaluate two within-layer nulls: within-layer time permutation and within-layer circular shift. We report mechanistic metrics (branching, $\Delta b$, $\chi$) and task metrics (perplexity, ARC accuracy) with bootstrap confidence intervals.

\section{Results}
\subsection{Event rasters and rate matching}
\Cref{fig:F01_RASTER_EXAMPLE} shows a representative token-by-layer event raster. Rate matching succeeds across all Dataset~A conditions, with maximum absolute per-layer rate error below the fixed tolerance (\Cref{fig:F02_RATE_MATCH_CHECK} and Table~\ref{tab:t01_selected}). This control is required to interpret any gain-dependent changes in branching.

\begin{figure*}[t]
  \centering
  \includegraphics[width=0.98\textwidth]{figures/fig_F01_RASTER_EXAMPLE.pdf}
  \caption{F01: Token-by-layer event raster example (Dataset A). Active sites are thresholded standardized MLP-gate activations; connected components correspond to avalanche-like event cascades on the token-by-layer lattice.}
  \label{fig:F01_RASTER_EXAMPLE}
\end{figure*}

\begin{figure}[t]
  \centering
  \includegraphics[width=\linewidth]{figures/fig_F02_RATE_MATCH_CHECK.pdf}
  \caption{F02: Rate-matching verification across gains (representative condition). Achieved marginal spike rates per layer match the target rate within tolerance across the full gain grid.}
  \label{fig:F02_RATE_MATCH_CHECK}
\end{figure}

\subsection{Gain affects branching under rate matching}
\Cref{fig:F03_BRANCHING_CURVES} plots $b_{\text{time}}$, $b_{\text{depth}}$, and $b_{\text{tot}}$ versus gain for both spike definitions at a representative target rate. Full results across all target rates are reported in Table~\ref{tab:t01_selected}. In the provided runs, $b_{\mathrm{tot}}$ varies with gain under rate matching (Table~\ref{tab:t01_selected}). The null-controlled residual $\Delta b_{\mathrm{tot}}$ remains non-zero under a strong within-layer permutation null (\Cref{fig:F04_NULL_DELTAB}; Table~\ref{tab:t01_selected}). This indicates connectivity structure beyond marginals even after rate matching.

\begin{figure*}[t]
  \centering
  \includegraphics[width=0.98\textwidth]{figures/fig_F03_BRANCHING_CURVES.pdf}
  \caption{F03: Directional branching metrics versus gain under rate matching (representative target rate). Total branching $b_{\mathrm{tot}}$ decomposes into token-direction and depth-direction components.}
  \label{fig:F03_BRANCHING_CURVES}
\end{figure*}

\begin{figure*}[t]
  \centering
  \includegraphics[width=0.98\textwidth]{figures/fig_F04_NULL_DELTAB.pdf}
  \caption{F04: Null-controlled residual connectivity $\Delta b$ relative to a within-layer time-permutation null that preserves per-layer event-count marginals exactly (representative target rate).}
  \label{fig:F04_NULL_DELTAB}
\end{figure*}

\subsection{Mechanistic \texorpdfstring{$g^{\star}$}{g*} selection}
\Cref{fig:F05_GSTAR_SELECTION} visualizes $g^{\star}$ selection by minimizing $|b_{\mathrm{tot}}(g)-1|$ on Dataset~A. The selected $g^{\star}$ differs across the eight (spike definition, target rate) conditions and can fall on the boundary of the gain grid (Table~\ref{tab:gstar}), illustrating why an explicit cross-dataset test is necessary.

\begin{table}[t]
\centering
\footnotesize
\setlength{\tabcolsep}{2.5pt}
\resizebox{\columnwidth}{!}{%
\begin{tabular}{llrrrrr}
\toprule
Spike def & Target rate & $g^{\star}$ & $b_{\mathrm{tot}}(g{=}1)$ & $b_{\mathrm{tot}}(g^{\star})$ & $\Delta b_{\mathrm{tot}}(g{=}1)$ & $\Delta b_{\mathrm{tot}}(g^{\star})$ \\
\midrule
 one-sided (+) & 1e-05 & 0.700 & 0.593 & 0.599 & 0.263 & 0.266 \\
 one-sided (+) & 2e-05 & 0.700 & 0.823 & 0.833 & 0.260 & 0.271 \\
 one-sided (+) & 4e-05 & 1.300 & 1.102 & 1.089 & 0.226 & 0.210 \\
 one-sided (+) & 8e-05 & 1.300 & 1.395 & 1.387 & 0.158 & 0.145 \\
 two-sided ($|\cdot|$) & 1e-05 & 0.850 & 0.627 & 0.629 & 0.295 & 0.300 \\
 two-sided ($|\cdot|$) & 2e-05 & 1.000 & 0.853 & 0.853 & 0.305 & 0.305 \\
 two-sided ($|\cdot|$) & 4e-05 & 1.300 & 1.113 & 1.103 & 0.267 & 0.253 \\
 two-sided ($|\cdot|$) & 8e-05 & 1.300 & 1.386 & 1.378 & 0.197 & 0.182 \\
\bottomrule
\end{tabular}%
}
\caption{Mechanistic gain calibration on Dataset A: selected $g^{\star}$ by minimizing $|b_{\mathrm{tot}}(g)-1|$ under rate-matched thresholds, and corresponding branching statistics at $g=1$ and $g^{\star}$. Values from Table~T01 and \texttt{gstar.json}.}
\label{tab:gstar}
\end{table}


\begin{figure}[t]
  \centering
  \includegraphics[width=\linewidth]{figures/fig_F05_GSTAR_SELECTION.pdf}
  \caption{F05: Mechanistic $g^{\star}$ selection by minimizing $|b_{\mathrm{tot}}(g)-1|$ on Dataset A (no performance signals used).}
  \label{fig:F05_GSTAR_SELECTION}
\end{figure}

\subsection{Cross-dataset evaluation: negative transfer}
We evaluate the mechanistically selected $g^{\star}$ unchanged on Dataset~B and ARC multiple-choice, comparing against $g{=}1$ (\Cref{fig:F06_GENERALIZATION_B,fig:F07_ARC_MCQ}). Figures visualize one representative condition with confidence intervals; Tables~\ref{tab:genB} and \ref{tab:arc} report all spike-definition/target-rate conditions with bootstrap confidence intervals. Replication across base vs instruct checkpoints is summarized in Appendix Table~\ref{tab:replication}.

\begin{table}[t]
\centering
\footnotesize
\setlength{\tabcolsep}{3pt}
\resizebox{\columnwidth}{!}{%
\begin{tabular}{llrrrr}
\toprule
 Spike def & Target rate & $g^{\star}$ & PPL($g{=}1$) [95\% CI] & PPL($g^{\star}$) [95\% CI] & $\Delta$PPL \\
\midrule
 one-sided (+) & 1e-05 & 0.700 & 15.173 [13.473, 17.130] & 17.627 [15.698, 19.816] & 2.453 \\
 one-sided (+) & 2e-05 & 0.700 & 15.173 [13.473, 17.130] & 17.627 [15.698, 19.816] & 2.453 \\
 one-sided (+) & 4e-05 & 1.300 & 15.173 [13.473, 17.130] & 16.300 [14.522, 18.308] & 1.127 \\
 one-sided (+) & 8e-05 & 1.300 & 15.173 [13.473, 17.130] & 16.300 [14.522, 18.308] & 1.127 \\
 two-sided ($|\cdot|$) & 1e-05 & 0.850 & 15.173 [13.473, 17.130] & 15.746 [13.973, 17.799] & 0.573 \\
 two-sided ($|\cdot|$) & 2e-05 & 1.000 & 15.173 [13.473, 17.130] & 15.173 [13.473, 17.130] & 0.000 \\
 two-sided ($|\cdot|$) & 4e-05 & 1.300 & 15.173 [13.473, 17.130] & 16.300 [14.522, 18.308] & 1.127 \\
 two-sided ($|\cdot|$) & 8e-05 & 1.300 & 15.173 [13.473, 17.130] & 16.300 [14.522, 18.308] & 1.127 \\
\bottomrule
\end{tabular}%
}
 \caption{Dataset B evaluation: perplexity at $g=1$ and at the mechanistically calibrated $g^{\star}$ (selected on Dataset A), with bootstrap 95\% confidence intervals from sequence resampling.}
\label{tab:genB}
\end{table}

\begin{table}[t]
\centering
\footnotesize
\setlength{\tabcolsep}{3pt}
\resizebox{\columnwidth}{!}{%
\begin{tabular}{llrrrr}
\toprule
 Spike def & Target rate & $g^{\star}$ & Acc($g{=}1$) [95\% CI] & Acc($g^{\star}$) [95\% CI] & $\Delta$Acc \\
\midrule
 one-sided (+) & 1e-05 & 0.700 & 0.885 [0.866, 0.902] & 0.873 [0.852, 0.890] & -0.012 \\
 one-sided (+) & 2e-05 & 0.700 & 0.885 [0.866, 0.902] & 0.873 [0.852, 0.890] & -0.012 \\
 one-sided (+) & 4e-05 & 1.300 & 0.885 [0.866, 0.902] & 0.857 [0.837, 0.875] & -0.028 \\
 one-sided (+) & 8e-05 & 1.300 & 0.885 [0.866, 0.902] & 0.857 [0.837, 0.875] & -0.028 \\
 two-sided ($|\cdot|$) & 1e-05 & 0.850 & 0.885 [0.866, 0.902] & 0.886 [0.866, 0.903] & 0.001 \\
 two-sided ($|\cdot|$) & 2e-05 & 1.000 & 0.885 [0.866, 0.902] & 0.885 [0.866, 0.902] & 0.000 \\
 two-sided ($|\cdot|$) & 4e-05 & 1.300 & 0.885 [0.866, 0.902] & 0.857 [0.837, 0.875] & -0.028 \\
 two-sided ($|\cdot|$) & 8e-05 & 1.300 & 0.885 [0.866, 0.902] & 0.857 [0.837, 0.875] & -0.028 \\
\bottomrule
\end{tabular}%
}
 \caption{ARC-Challenge multiple-choice evaluation: accuracy at $g=1$ vs $g^{\star}$ (selected on Dataset A), with bootstrap 95\% confidence intervals from question resampling.}
\label{tab:arc}
\end{table}


\begin{figure}[t]
  \centering
  \includegraphics[width=\linewidth]{figures/fig_F06_GENERALIZATION_B.pdf}
  \caption{F06: Dataset B evaluation at $g{=}1$ vs $g^{\star}$ (selected on Dataset A), representative condition with confidence intervals. Tables~\ref{tab:genB} and \ref{tab:t01_selected} report all conditions.}
  \label{fig:F06_GENERALIZATION_B}
\end{figure}

\begin{figure}[t]
  \centering
  \includegraphics[width=\linewidth]{figures/fig_F07_ARC_MCQ.pdf}
  \caption{F07: ARC multiple-choice accuracy at $g{=}1$ vs $g^{\star}$ with confidence intervals (representative condition). Table~\ref{tab:arc} reports all conditions.}
  \label{fig:F07_ARC_MCQ}
\end{figure}

\subsection{Robustness across spike definitions}
\Cref{fig:F08_SPIKEDEF_ROBUST} summarizes qualitative robustness across spike definitions, reducing the risk that results depend on a particular thresholding convention.

\begin{figure}[t]
  \centering
  \includegraphics[width=\linewidth]{figures/fig_F08_SPIKEDEF_ROBUST.pdf}
  \caption{F08: Robustness across spike definitions (representative target rate). Qualitative gain-dependent patterns persist under one-sided and two-sided event definitions.}
  \label{fig:F08_SPIKEDEF_ROBUST}
\end{figure}

\subsection{Claim boundaries}
We do not infer criticality from tail shapes alone. Tail fits are reported as descriptive diagnostics \citep{clauset2009power,touboul2010can}, and the signature suite is treated as a set of falsifiable probes that can yield negative results under the same controls.

\section{Discussion, limitations, and ethics}
Our results support two practical takeaways. First, gain scaling changes local event connectivity even when marginal rates are matched and marginals are controlled by a strong null. Second, a mechanistic calibration based on a branching target does not necessarily generalize to task metrics; negative transfer should be expected and reported.

\textbf{Limitations.} The included runs use deterministic dataset slices to fit within a single-GPU budget (Dataset~A: 64 calibration sequences of length 256 for mechanistic metrics; plus a separate 128-sequence slice for raster/null extraction; Dataset~B: 96 sequences of length 256; ARC-Challenge test split). Uncertainty estimates for mechanistic metrics can still tighten with additional sampling. We evaluate two checkpoints (instruction-tuned and base) within one model family and a single gain intervention family.

\textbf{Ethics.} This work analyzes internal activations of open models and does not introduce new training data or deployment. Interpretability results should not be overinterpreted as cognitive equivalence.

\section*{Reproducibility statement}
All figures and tables referenced in this paper are included as immutable artifacts in the accompanying bundle. Each producing run contains \texttt{run\_\allowbreak record.json} and \texttt{config\_\allowbreak resolved.yaml} describing model, data slices, conditions, and artifact hashes. Appendix~\ref{sec:provenance} lists the run identifiers used for each main figure and table.

\clearpage
\begin{thebibliography}{9}
\small
\raggedright

\bibitem[Beggs and Plenz(2003)]{beggs2003neuronal}
John M. Beggs and Dietmar Plenz.
\newblock Neuronal avalanches in neocortical circuits.
\newblock \emph{Journal of Neuroscience}, 23(35):11167--11177, 2003.
\newblock doi:10.1523/JNEUROSCI.23-35-11167.2003.

\bibitem[Clauset et~al.(2009)Clauset, Shalizi, and Newman]{clauset2009power}
Aaron Clauset, Cosma Rohilla Shalizi, and Mark E.~J. Newman.
\newblock Power-law distributions in empirical data.
\newblock \emph{SIAM Review}, 51(4):661--703, 2009.
\newblock doi:10.1137/070710111.

\bibitem[Touboul and Destexhe(2010)]{touboul2010can}
Jonathan Touboul and Alain Destexhe.
\newblock Can power-law scaling and neuronal avalanches arise from stochastic dynamics?
\newblock \emph{PLOS ONE}, 5(2):e8982, 2010.
\newblock doi:10.1371/journal.pone.0008982.

\bibitem[Sun et~al.(2024)Sun, Chen, Kolter, and Liu]{sun2024massive}
Mingjie Sun, Xinlei Chen, J. Zico Kolter, and Zhuang Liu.
\newblock Massive activations in large language models.
\newblock arXiv preprint arXiv:2402.17762, 2024.
\newblock doi:10.48550/arXiv.2402.17762.

\bibitem[Schoenholz et~al.(2016)Schoenholz, Gilmer, Ganguli, and Sohl-Dickstein]{schoenholz2016deep}
Samuel S. Schoenholz, Justin Gilmer, Surya Ganguli, and Jascha Sohl-Dickstein.
\newblock Deep information propagation.
\newblock arXiv preprint arXiv:1611.01232, 2016.
\newblock doi:10.48550/arXiv.1611.01232.

\bibitem[Poole et~al.(2016)Poole, Lahiri, Raghu, Sohl-Dickstein, and Ganguli]{poole2016exponential}
Ben Poole, Subhaneil Lahiri, Maithra Raghu, Jascha Sohl-Dickstein, and Surya Ganguli.
\newblock Exponential expressivity in deep neural networks through transient chaos.
\newblock arXiv preprint arXiv:1606.05340, 2016.
\newblock doi:10.48550/arXiv.1606.05340.

\bibitem[Pennington et~al.(2017)Pennington, Schoenholz, and Ganguli]{pennington2017resurrecting}
Jeffrey Pennington, Samuel S. Schoenholz, and Surya Ganguli.
\newblock Resurrecting the sigmoid in deep learning through dynamical isometry: theory and practice.
\newblock arXiv preprint arXiv:1711.04735, 2017.
\newblock doi:10.48550/arXiv.1711.04735.

\bibitem[Shazeer(2020)]{shazeer2020glu}
Noam Shazeer.
\newblock GLU variants improve transformer.
\newblock arXiv preprint arXiv:2002.05202, 2020.
\newblock doi:10.48550/arXiv.2002.05202.

\end{thebibliography}

\clearpage
\appendix
\onecolumn
\section{Appendix: Additional tables, figures, and provenance}
\label{sec:provenance}

\subsection{Susceptibility curves}
\begin{figure}[H]
  \centering
  \includegraphics[width=\linewidth]{figures/appendix/fig_F09_CHI_CURVES}
  \caption{F09: Susceptibility proxy $\chi(g)$ with uncertainty across spike definitions (representative target rate). Full results are in Table~T01.}
  \label{fig:F09_CHI_CURVES}
\end{figure}

\subsection{Null comparison}
\begin{figure}[H]
  \centering
  \includegraphics[width=\linewidth]{figures/appendix/fig_F10_NULL_COMPARE}
  \caption{F10: $\Delta b_{\mathrm{tot}}(g)$ under multiple nulls, including a structure-preserving circular-shift null.}
  \label{fig:F10_NULL_COMPARE}
\end{figure}

\subsection{Ablations}
% NOTE: This table is typeset as a longtable to avoid float-too-large warnings in the appendix.
\begin{center}
\scriptsize
\setlength{\tabcolsep}{3pt}
\renewcommand{\arraystretch}{1.05}
\setlength{\LTpre}{0pt}
\setlength{\LTpost}{0pt}
\begin{longtable}{llrrrr}
\caption{Ablation comparison of gain interventions at matched rates.}\label{tab:ablations}\\
\toprule
Intervention & Spike def & Target rate & $g$ & $\Delta b_{\mathrm{tot}}$ & max|rate err| \\
\midrule
\endfirsthead
\multicolumn{6}{l}{\textbf{Table \thetable} (continued)}\\
\toprule
Intervention & Spike def & Target rate & $g$ & $\Delta b_{\mathrm{tot}}$ & max|rate err| \\
\midrule
\endhead
\midrule
\multicolumn{6}{r}{\footnotesize Continued on next page}\\
\endfoot
\bottomrule
\endlastfoot
\texttt{ATTN\_GLOBAL} & one-sided (+) & 1e-05 & 0.700 & 0.239 & 9.81e-08 \\
 \texttt{ATTN\_GLOBAL} & one-sided (+) & 2e-05 & 0.700 & 0.242 & 2.53e-07 \\
 \texttt{ATTN\_GLOBAL} & one-sided (+) & 4e-05 & 1.300 & 0.235 & 3.86e-07 \\
 \texttt{ATTN\_GLOBAL} & one-sided (+) & 8e-05 & 1.300 & 0.164 & 9.53e-07 \\
 \texttt{ATTN\_GLOBAL} & two-sided ($|\cdot|$) & 1e-05 & 0.850 & 0.287 & 4.34e-08 \\
 \texttt{ATTN\_GLOBAL} & two-sided ($|\cdot|$) & 2e-05 & 1.000 & 0.301 & 6.21e-08 \\
 \texttt{ATTN\_GLOBAL} & two-sided ($|\cdot|$) & 4e-05 & 1.300 & 0.273 & 2.06e-07 \\
 \texttt{ATTN\_GLOBAL} & two-sided ($|\cdot|$) & 8e-05 & 1.300 & 0.201 & 4.36e-07 \\
 \texttt{MLP\_BAND\_EARLY} & one-sided (+) & 1e-05 & 0.700 & 0.256 & 1.03e-07 \\
 \texttt{MLP\_BAND\_EARLY} & one-sided (+) & 2e-05 & 0.700 & 0.254 & 2.12e-07 \\
 \texttt{MLP\_BAND\_EARLY} & one-sided (+) & 4e-05 & 1.300 & 0.220 & 4.85e-07 \\
 \texttt{MLP\_BAND\_EARLY} & one-sided (+) & 8e-05 & 1.300 & 0.152 & 1.07e-06 \\
 \texttt{MLP\_BAND\_EARLY} & two-sided ($|\cdot|$) & 1e-05 & 0.850 & 0.296 & 4.36e-08 \\
 \texttt{MLP\_BAND\_EARLY} & two-sided ($|\cdot|$) & 2e-05 & 1.000 & 0.306 & 6.21e-08 \\
 \texttt{MLP\_BAND\_EARLY} & two-sided ($|\cdot|$) & 4e-05 & 1.300 & 0.262 & 1.93e-07 \\
 \texttt{MLP\_BAND\_EARLY} & two-sided ($|\cdot|$) & 8e-05 & 1.300 & 0.188 & 4.27e-07 \\
 \texttt{MLP\_BAND\_LATE} & one-sided (+) & 1e-05 & 0.700 & 0.262 & 1.19e-07 \\
 \texttt{MLP\_BAND\_LATE} & one-sided (+) & 2e-05 & 0.700 & 0.260 & 1.93e-07 \\
 \texttt{MLP\_BAND\_LATE} & one-sided (+) & 4e-05 & 1.300 & 0.223 & 3.66e-07 \\
 \texttt{MLP\_BAND\_LATE} & one-sided (+) & 8e-05 & 1.300 & 0.156 & 8.43e-07 \\
 \texttt{MLP\_BAND\_LATE} & two-sided ($|\cdot|$) & 1e-05 & 0.850 & 0.298 & 3.65e-08 \\
 \texttt{MLP\_BAND\_LATE} & two-sided ($|\cdot|$) & 2e-05 & 1.000 & 0.303 & 6.21e-08 \\
 \texttt{MLP\_BAND\_LATE} & two-sided ($|\cdot|$) & 4e-05 & 1.300 & 0.261 & 2.05e-07 \\
 \texttt{MLP\_BAND\_LATE} & two-sided ($|\cdot|$) & 8e-05 & 1.300 & 0.192 & 4.44e-07 \\
 \texttt{MLP\_BAND\_MID} & one-sided (+) & 1e-05 & 0.700 & 0.263 & 9.37e-08 \\
 \texttt{MLP\_BAND\_MID} & one-sided (+) & 2e-05 & 0.700 & 0.263 & 2.62e-07 \\
 \texttt{MLP\_BAND\_MID} & one-sided (+) & 4e-05 & 1.300 & 0.218 & 8.40e-07 \\
 \texttt{MLP\_BAND\_MID} & one-sided (+) & 8e-05 & 1.300 & 0.154 & 9.60e-07 \\
 \texttt{MLP\_BAND\_MID} & two-sided ($|\cdot|$) & 1e-05 & 0.850 & 0.297 & 5.15e-08 \\
 \texttt{MLP\_BAND\_MID} & two-sided ($|\cdot|$) & 2e-05 & 1.000 & 0.304 & 6.21e-08 \\
 \texttt{MLP\_BAND\_MID} & two-sided ($|\cdot|$) & 4e-05 & 1.300 & 0.262 & 2.62e-07 \\
 \texttt{MLP\_BAND\_MID} & two-sided ($|\cdot|$) & 8e-05 & 1.300 & 0.191 & 4.34e-07 \\
 \texttt{MLP\_GLOBAL} & one-sided (+) & 1e-05 & 1.000 & 0.263 & 1.12e-07 \\
 \texttt{MLP\_GLOBAL} & one-sided (+) & 1e-05 & 0.700 & 0.266 & 9.63e-08 \\
 \texttt{MLP\_GLOBAL} & one-sided (+) & 2e-05 & 1.000 & 0.260 & 1.71e-07 \\
 \texttt{MLP\_GLOBAL} & one-sided (+) & 2e-05 & 0.700 & 0.271 & 1.78e-07 \\
 \texttt{MLP\_GLOBAL} & one-sided (+) & 4e-05 & 1.000 & 0.226 & 3.66e-07 \\
 \texttt{MLP\_GLOBAL} & one-sided (+) & 4e-05 & 1.300 & 0.210 & 4.65e-07 \\
 \texttt{MLP\_GLOBAL} & one-sided (+) & 8e-05 & 1.000 & 0.158 & 8.98e-07 \\
 \texttt{MLP\_GLOBAL} & one-sided (+) & 8e-05 & 1.300 & 0.145 & 1.12e-06 \\
 \texttt{MLP\_GLOBAL} & two-sided ($|\cdot|$) & 1e-05 & 1.000 & 0.295 & 3.65e-08 \\
 \texttt{MLP\_GLOBAL} & two-sided ($|\cdot|$) & 1e-05 & 0.850 & 0.300 & 4.30e-08 \\
 \texttt{MLP\_GLOBAL} & two-sided ($|\cdot|$) & 2e-05 & 1.000 & 0.305 & 6.21e-08 \\
 \texttt{MLP\_GLOBAL} & two-sided ($|\cdot|$) & 2e-05 & 1.000 & 0.305 & 6.21e-08 \\
 \texttt{MLP\_GLOBAL} & two-sided ($|\cdot|$) & 4e-05 & 1.000 & 0.267 & 2.05e-07 \\
 \texttt{MLP\_GLOBAL} & two-sided ($|\cdot|$) & 4e-05 & 1.300 & 0.253 & 2.54e-07 \\
 \texttt{MLP\_GLOBAL} & two-sided ($|\cdot|$) & 8e-05 & 1.000 & 0.197 & 4.44e-07 \\
 \texttt{MLP\_GLOBAL} & two-sided ($|\cdot|$) & 8e-05 & 1.300 & 0.182 & 6.07e-07 \\
\end{longtable}
\end{center}

\begin{figure}[H]
  \centering
  \includegraphics[width=\linewidth]{figures/appendix/fig_F11_ABLATIONS}
  \caption{F11: Ablation comparison of gain interventions (MLP vs attention and layer-banded variants) at matched rates.}
  \label{fig:F11_ABLATIONS}
\end{figure}

\subsection{Tail fits and crackling diagnostics (descriptive)}
% NOTE: This table is typeset as a longtable to avoid float-too-large warnings in the appendix.
\begin{center}
\scriptsize
\setlength{\tabcolsep}{2pt}
\renewcommand{\arraystretch}{1.05}
\setlength{\LTpre}{0pt}
\setlength{\LTpost}{0pt}
\begin{longtable}{llrrrrrr}
\caption{Tail-fit diagnostics on avalanche size (descriptive only). Continuous-approximation fits on the upper tail defined by a fixed percentile.}\label{tab:tail_fits}\\
\toprule
Spike def & Target rate & $g$ & $x_{\min}$ & $n_{\mathrm{tail}}$ & $\alpha$ (PL) & LLR(PL--LN) & LLR(PL--EXP) \\
\midrule
\endfirsthead
\multicolumn{8}{l}{\textbf{Table \thetable} (continued)}\\
\toprule
Spike def & Target rate & $g$ & $x_{\min}$ & $n_{\mathrm{tail}}$ & $\alpha$ (PL) & LLR(PL--LN) & LLR(PL--EXP) \\
\midrule
\endhead
\midrule
\multicolumn{8}{r}{\footnotesize Continued on next page}\\
\endfoot
\bottomrule
\endlastfoot
one-sided (+) & 1e-05 & 0.700 & 3.000 & 7948 & 2.669 & 4056.992 & 2297.358 \\
 one-sided (+) & 1e-05 & 0.800 & 3.000 & 7976 & 2.668 & 3987.043 & 2252.031 \\
 one-sided (+) & 1e-05 & 0.850 & 3.000 & 8000 & 2.676 & 4029.989 & 2270.047 \\
 one-sided (+) & 1e-05 & 0.900 & 3.000 & 7969 & 2.675 & 4032.787 & 2269.391 \\
 one-sided (+) & 1e-05 & 0.950 & 3.000 & 8000 & 2.682 & 4075.005 & 2319.080 \\
 one-sided (+) & 1e-05 & 1.000 & 3.000 & 7981 & 2.685 & 4090.435 & 2348.173 \\
 one-sided (+) & 1e-05 & 1.050 & 3.000 & 7948 & 2.676 & 4071.773 & 2345.749 \\
 one-sided (+) & 1e-05 & 1.100 & 3.000 & 7933 & 2.683 & 4089.890 & 2371.667 \\
 one-sided (+) & 1e-05 & 1.150 & 3.000 & 7979 & 2.696 & 4147.541 & 2421.596 \\
 one-sided (+) & 1e-05 & 1.200 & 3.000 & 7932 & 2.705 & 4178.151 & 2508.271 \\
 one-sided (+) & 1e-05 & 1.300 & 3.000 & 7886 & 2.711 & 4181.969 & 2579.957 \\
 one-sided (+) & 2e-05 & 0.700 & 5.000 & 7017 & 2.203 & 2814.018 & 2419.748 \\
 one-sided (+) & 2e-05 & 0.800 & 5.000 & 7036 & 2.206 & 2809.992 & 2411.345 \\
 one-sided (+) & 2e-05 & 0.850 & 5.000 & 7036 & 2.208 & 2806.333 & 2434.081 \\
 one-sided (+) & 2e-05 & 0.900 & 5.000 & 7030 & 2.207 & 2802.220 & 2425.993 \\
 one-sided (+) & 2e-05 & 0.950 & 5.000 & 7054 & 2.205 & 2773.071 & 2361.295 \\
 one-sided (+) & 2e-05 & 1.000 & 5.000 & 7052 & 2.198 & 2729.209 & 2298.310 \\
 one-sided (+) & 2e-05 & 1.050 & 5.000 & 7045 & 2.198 & 2732.065 & 2316.180 \\
 one-sided (+) & 2e-05 & 1.100 & 5.000 & 7057 & 2.205 & 2788.389 & 2366.974 \\
 one-sided (+) & 2e-05 & 1.150 & 5.000 & 7107 & 2.210 & 2804.680 & 2369.985 \\
 one-sided (+) & 2e-05 & 1.200 & 5.000 & 7066 & 2.206 & 2760.866 & 2339.463 \\
 one-sided (+) & 2e-05 & 1.300 & 5.000 & 7058 & 2.218 & 2821.391 & 2445.851 \\
 one-sided (+) & 4e-05 & 0.700 & 7.000 & 5438 & 1.853 & 2053.212 & 3350.284 \\
 one-sided (+) & 4e-05 & 0.800 & 7.000 & 5480 & 1.851 & 2036.804 & 3298.712 \\
 one-sided (+) & 4e-05 & 0.850 & 7.000 & 5468 & 1.850 & 2036.417 & 3291.294 \\
 one-sided (+) & 4e-05 & 0.900 & 7.000 & 5498 & 1.853 & 2049.012 & 3321.405 \\
 one-sided (+) & 4e-05 & 0.950 & 7.000 & 5532 & 1.856 & 2090.317 & 3336.432 \\
 one-sided (+) & 4e-05 & 1.000 & 7.000 & 5485 & 1.856 & 2091.606 & 3366.843 \\
 one-sided (+) & 4e-05 & 1.050 & 7.000 & 5464 & 1.853 & 2073.021 & 3330.642 \\
 one-sided (+) & 4e-05 & 1.100 & 7.000 & 5450 & 1.848 & 2008.264 & 3248.724 \\
 one-sided (+) & 4e-05 & 1.150 & 7.000 & 5512 & 1.855 & 2065.282 & 3309.023 \\
 one-sided (+) & 4e-05 & 1.200 & 7.000 & 5487 & 1.851 & 2038.951 & 3257.426 \\
 one-sided (+) & 4e-05 & 1.300 & 7.000 & 5508 & 1.847 & 1992.441 & 3196.796 \\
 one-sided (+) & 8e-05 & 0.700 & 8.000 & 2480 & 1.667 & 1320.508 & 3931.708 \\
 one-sided (+) & 8e-05 & 0.800 & 8.000 & 2599 & 1.658 & 1349.754 & 3908.835 \\
 one-sided (+) & 8e-05 & 0.850 & 8.000 & 2615 & 1.651 & 1329.981 & 3847.509 \\
 one-sided (+) & 8e-05 & 0.900 & 8.000 & 2577 & 1.647 & 1313.601 & 3788.598 \\
 one-sided (+) & 8e-05 & 0.950 & 8.000 & 2576 & 1.643 & 1293.005 & 3745.451 \\
 one-sided (+) & 8e-05 & 1.000 & 8.000 & 2621 & 1.660 & 1373.614 & 3933.414 \\
 one-sided (+) & 8e-05 & 1.050 & 8.000 & 2597 & 1.665 & 1378.840 & 3976.578 \\
 one-sided (+) & 8e-05 & 1.100 & 8.000 & 2618 & 1.667 & 1398.571 & 3997.533 \\
 one-sided (+) & 8e-05 & 1.150 & 8.000 & 2644 & 1.681 & 1451.381 & 4156.312 \\
 one-sided (+) & 8e-05 & 1.200 & 8.000 & 2641 & 1.683 & 1464.244 & 4170.795 \\
 one-sided (+) & 8e-05 & 1.300 & 7.000 & 2924 & 1.699 & 1678.102 & 4884.305 \\
 two-sided ($|\cdot|$) & 1e-05 & 0.700 & 3.000 & 7602 & 2.588 & 3954.727 & 2594.214 \\
 two-sided ($|\cdot|$) & 1e-05 & 0.800 & 3.000 & 7594 & 2.585 & 3967.263 & 2579.633 \\
 two-sided ($|\cdot|$) & 1e-05 & 0.850 & 3.000 & 7595 & 2.576 & 3948.046 & 2542.628 \\
 two-sided ($|\cdot|$) & 1e-05 & 0.900 & 3.000 & 7621 & 2.569 & 3877.639 & 2462.573 \\
 two-sided ($|\cdot|$) & 1e-05 & 0.950 & 3.000 & 7615 & 2.560 & 3810.091 & 2383.591 \\
 two-sided ($|\cdot|$) & 1e-05 & 1.000 & 3.000 & 7641 & 2.562 & 3816.369 & 2371.430 \\
 two-sided ($|\cdot|$) & 1e-05 & 1.050 & 3.000 & 7685 & 2.571 & 3871.749 & 2377.783 \\
 two-sided ($|\cdot|$) & 1e-05 & 1.100 & 3.000 & 7685 & 2.575 & 3876.614 & 2385.655 \\
 two-sided ($|\cdot|$) & 1e-05 & 1.150 & 3.000 & 7713 & 2.577 & 3884.445 & 2372.329 \\
 two-sided ($|\cdot|$) & 1e-05 & 1.200 & 3.000 & 7644 & 2.566 & 3819.534 & 2368.780 \\
 two-sided ($|\cdot|$) & 1e-05 & 1.300 & 3.000 & 7582 & 2.560 & 3774.944 & 2392.893 \\
 two-sided ($|\cdot|$) & 2e-05 & 0.700 & 5.000 & 6804 & 2.172 & 2729.655 & 2542.164 \\
 two-sided ($|\cdot|$) & 2e-05 & 0.800 & 5.000 & 6704 & 2.157 & 2648.830 & 2464.156 \\
 two-sided ($|\cdot|$) & 2e-05 & 0.850 & 5.000 & 6786 & 2.166 & 2719.956 & 2514.505 \\
 two-sided ($|\cdot|$) & 2e-05 & 0.900 & 5.000 & 6751 & 2.162 & 2677.362 & 2487.523 \\
 two-sided ($|\cdot|$) & 2e-05 & 0.950 & 5.000 & 6777 & 2.169 & 2738.187 & 2553.888 \\
 two-sided ($|\cdot|$) & 2e-05 & 1.000 & 5.000 & 6717 & 2.157 & 2653.084 & 2481.295 \\
 two-sided ($|\cdot|$) & 2e-05 & 1.050 & 5.000 & 6653 & 2.151 & 2597.878 & 2466.324 \\
 two-sided ($|\cdot|$) & 2e-05 & 1.100 & 5.000 & 6691 & 2.150 & 2612.198 & 2434.546 \\
 two-sided ($|\cdot|$) & 2e-05 & 1.150 & 5.000 & 6666 & 2.145 & 2578.680 & 2394.654 \\
 two-sided ($|\cdot|$) & 2e-05 & 1.200 & 5.000 & 6681 & 2.151 & 2581.616 & 2427.482 \\
 two-sided ($|\cdot|$) & 2e-05 & 1.300 & 5.000 & 6628 & 2.151 & 2568.994 & 2460.674 \\
 two-sided ($|\cdot|$) & 4e-05 & 0.700 & 7.000 & 5408 & 1.831 & 1934.127 & 3126.533 \\
 two-sided ($|\cdot|$) & 4e-05 & 0.800 & 7.000 & 5424 & 1.830 & 1909.250 & 3094.482 \\
 two-sided ($|\cdot|$) & 4e-05 & 0.850 & 7.000 & 5480 & 1.833 & 1920.907 & 3109.686 \\
 two-sided ($|\cdot|$) & 4e-05 & 0.900 & 7.000 & 5432 & 1.836 & 1941.444 & 3178.971 \\
 two-sided ($|\cdot|$) & 4e-05 & 0.950 & 7.000 & 5445 & 1.839 & 1957.922 & 3213.232 \\
 two-sided ($|\cdot|$) & 4e-05 & 1.000 & 7.000 & 5402 & 1.833 & 1924.979 & 3140.572 \\
 two-sided ($|\cdot|$) & 4e-05 & 1.050 & 7.000 & 5390 & 1.835 & 1923.012 & 3170.906 \\
 two-sided ($|\cdot|$) & 4e-05 & 1.100 & 7.000 & 5366 & 1.830 & 1908.787 & 3124.674 \\
 two-sided ($|\cdot|$) & 4e-05 & 1.150 & 7.000 & 5341 & 1.831 & 1921.037 & 3146.645 \\
 two-sided ($|\cdot|$) & 4e-05 & 1.200 & 7.000 & 5321 & 1.822 & 1841.185 & 3016.642 \\
 two-sided ($|\cdot|$) & 4e-05 & 1.300 & 7.000 & 5400 & 1.836 & 1930.342 & 3160.118 \\
 two-sided ($|\cdot|$) & 8e-05 & 0.700 & 9.000 & 2601 & 1.610 & 1127.714 & 3086.452 \\
 two-sided ($|\cdot|$) & 8e-05 & 0.800 & 9.000 & 2661 & 1.607 & 1128.619 & 3053.789 \\
 two-sided ($|\cdot|$) & 8e-05 & 0.850 & 9.000 & 2686 & 1.610 & 1155.508 & 3096.750 \\
 two-sided ($|\cdot|$) & 8e-05 & 0.900 & 9.000 & 2684 & 1.606 & 1132.866 & 3047.846 \\
 two-sided ($|\cdot|$) & 8e-05 & 0.950 & 9.000 & 2705 & 1.608 & 1145.213 & 3070.833 \\
 two-sided ($|\cdot|$) & 8e-05 & 1.000 & 9.000 & 2686 & 1.608 & 1147.021 & 3075.422 \\
 two-sided ($|\cdot|$) & 8e-05 & 1.050 & 9.000 & 2712 & 1.615 & 1190.946 & 3158.379 \\
 two-sided ($|\cdot|$) & 8e-05 & 1.100 & 9.000 & 2711 & 1.614 & 1188.181 & 3142.813 \\
 two-sided ($|\cdot|$) & 8e-05 & 1.150 & 9.000 & 2716 & 1.622 & 1206.134 & 3234.563 \\
 two-sided ($|\cdot|$) & 8e-05 & 1.200 & 9.000 & 2742 & 1.619 & 1205.001 & 3199.957 \\
 two-sided ($|\cdot|$) & 8e-05 & 1.300 & 8.000 & 2949 & 1.642 & 1402.427 & 3859.723 \\
\end{longtable}
\end{center}

% NOTE: This table is typeset as a longtable to avoid float-too-large warnings in the appendix.
\begin{center}
\scriptsize
\setlength{\tabcolsep}{2pt}
\renewcommand{\arraystretch}{1.05}
\setlength{\LTpre}{0pt}
\setlength{\LTpost}{0pt}
\begin{longtable}{llrrrrrrrr}
\caption{Crackling fit diagnostics (descriptive) with a fail-closed gate.}\label{tab:crackling}\\
\toprule
Spike def & Target rate & $g$ & $\gamma$ & CI low & CI high & CI width & $n_{\mathrm{avals}}$ & $R^2$ & gate \\
\midrule
\endfirsthead
\multicolumn{10}{l}{\textbf{Table \thetable} (continued)}\\
\toprule
Spike def & Target rate & $g$ & $\gamma$ & CI low & CI high & CI width & $n_{\mathrm{avals}}$ & $R^2$ & gate \\
\midrule
\endhead
\midrule
\multicolumn{10}{r}{\footnotesize Continued on next page}\\
\endfoot
\bottomrule
\endlastfoot
one-sided (+) & 1e-05 & 0.700 & 1.745 & 1.631 & 1.877 & 0.246 & 2969 & 0.970 & pass \\
 one-sided (+) & 1e-05 & 0.800 & 1.718 & 1.630 & 1.896 & 0.266 & 2911 & 0.943 & pass \\
 one-sided (+) & 1e-05 & 0.850 & 1.814 & 1.663 & 1.912 & 0.249 & 2856 & 0.970 & pass \\
 one-sided (+) & 1e-05 & 0.900 & 1.793 & 1.602 & 1.953 & 0.352 & 2801 & 0.930 & pass \\
 one-sided (+) & 1e-05 & 0.950 & 1.792 & 1.632 & 1.903 & 0.272 & 2756 & 0.970 & pass \\
 one-sided (+) & 1e-05 & 1.000 & 1.686 & 1.568 & 1.897 & 0.329 & 2719 & 0.942 & pass \\
 one-sided (+) & 1e-05 & 1.050 & 1.682 & 1.559 & 1.912 & 0.352 & 2701 & 0.957 & pass \\
 one-sided (+) & 1e-05 & 1.100 & 1.632 & 1.471 & 1.859 & 0.387 & 2685 & 0.949 & pass \\
 one-sided (+) & 1e-05 & 1.150 & 1.702 & 1.479 & 1.834 & 0.355 & 2707 & 0.970 & pass \\
 one-sided (+) & 1e-05 & 1.200 & 1.730 & 1.591 & 1.858 & 0.267 & 2653 & 0.975 & pass \\
 one-sided (+) & 1e-05 & 1.300 & 1.702 & 1.574 & 1.847 & 0.273 & 2606 & 0.977 & pass \\
 one-sided (+) & 2e-05 & 0.700 & 1.726 & 1.649 & 1.835 & 0.186 & 5095 & 0.981 & pass \\
 one-sided (+) & 2e-05 & 0.800 & 1.729 & 1.650 & 1.818 & 0.168 & 5059 & 0.972 & pass \\
 one-sided (+) & 2e-05 & 0.850 & 1.751 & 1.686 & 1.817 & 0.130 & 5064 & 0.981 & pass \\
 one-sided (+) & 2e-05 & 0.900 & 1.708 & 1.635 & 1.806 & 0.171 & 5030 & 0.943 & pass \\
 one-sided (+) & 2e-05 & 0.950 & 1.598 & 1.533 & 1.806 & 0.274 & 5031 & 0.906 & pass \\
 one-sided (+) & 2e-05 & 1.000 & 1.613 & 1.526 & 1.832 & 0.306 & 4995 & 0.902 & pass \\
 one-sided (+) & 2e-05 & 1.050 & 1.670 & 1.589 & 1.846 & 0.257 & 4986 & 0.932 & pass \\
 one-sided (+) & 2e-05 & 1.100 & 1.686 & 1.609 & 1.845 & 0.236 & 4978 & 0.932 & pass \\
 one-sided (+) & 2e-05 & 1.150 & 1.729 & 1.624 & 1.820 & 0.196 & 5005 & 0.971 & pass \\
 one-sided (+) & 2e-05 & 1.200 & 1.565 & 1.503 & 1.762 & 0.259 & 4994 & 0.913 & pass \\
 one-sided (+) & 2e-05 & 1.300 & 1.685 & 1.608 & 1.757 & 0.149 & 4927 & 0.972 & pass \\
 one-sided (+) & 4e-05 & 0.700 & 1.652 & 1.616 & 1.711 & 0.095 & 5081 & 0.967 & pass \\
 one-sided (+) & 4e-05 & 0.800 & 1.684 & 1.650 & 1.744 & 0.094 & 5066 & 0.984 & pass \\
 one-sided (+) & 4e-05 & 0.850 & 1.675 & 1.653 & 1.738 & 0.084 & 5049 & 0.981 & pass \\
 one-sided (+) & 4e-05 & 0.900 & 1.683 & 1.657 & 1.737 & 0.081 & 5092 & 0.975 & pass \\
 one-sided (+) & 4e-05 & 0.950 & 1.639 & 1.616 & 1.722 & 0.107 & 5114 & 0.966 & pass \\
 one-sided (+) & 4e-05 & 1.000 & 1.669 & 1.625 & 1.730 & 0.105 & 5096 & 0.981 & pass \\
 one-sided (+) & 4e-05 & 1.050 & 1.650 & 1.616 & 1.713 & 0.096 & 5096 & 0.966 & pass \\
 one-sided (+) & 4e-05 & 1.100 & 1.602 & 1.579 & 1.693 & 0.114 & 5109 & 0.960 & pass \\
 one-sided (+) & 4e-05 & 1.150 & 1.624 & 1.599 & 1.693 & 0.094 & 5180 & 0.973 & pass \\
 one-sided (+) & 4e-05 & 1.200 & 1.645 & 1.625 & 1.699 & 0.074 & 5146 & 0.976 & pass \\
 one-sided (+) & 4e-05 & 1.300 & 1.647 & 1.622 & 1.712 & 0.090 & 5226 & 0.984 & pass \\
 one-sided (+) & 8e-05 & 0.700 & 1.685 & 1.654 & 1.739 & 0.085 & 2272 & 0.956 & pass \\
 one-sided (+) & 8e-05 & 0.800 & 1.736 & 1.709 & 1.788 & 0.079 & 2371 & 0.969 & pass \\
 one-sided (+) & 8e-05 & 0.850 & 1.703 & 1.670 & 1.753 & 0.083 & 2397 & 0.966 & pass \\
 one-sided (+) & 8e-05 & 0.900 & 1.725 & 1.690 & 1.773 & 0.082 & 2386 & 0.963 & pass \\
 one-sided (+) & 8e-05 & 0.950 & 1.705 & 1.685 & 1.764 & 0.079 & 2398 & 0.960 & pass \\
 one-sided (+) & 8e-05 & 1.000 & 1.717 & 1.688 & 1.774 & 0.085 & 2420 & 0.963 & pass \\
 one-sided (+) & 8e-05 & 1.050 & 1.712 & 1.683 & 1.765 & 0.082 & 2416 & 0.964 & pass \\
 one-sided (+) & 8e-05 & 1.100 & 1.709 & 1.678 & 1.754 & 0.076 & 2466 & 0.961 & pass \\
 one-sided (+) & 8e-05 & 1.150 & 1.698 & 1.674 & 1.761 & 0.087 & 2498 & 0.958 & pass \\
 one-sided (+) & 8e-05 & 1.200 & 1.708 & 1.673 & 1.756 & 0.083 & 2544 & 0.955 & pass \\
 one-sided (+) & 8e-05 & 1.300 & 1.727 & 1.694 & 1.791 & 0.097 & 2548 & 0.962 & pass \\
 two-sided ($|\cdot|$) & 1e-05 & 0.700 & 1.736 & 1.637 & 1.834 & 0.197 & 2587 & 0.945 & pass \\
 two-sided ($|\cdot|$) & 1e-05 & 0.800 & 1.719 & 1.585 & 1.879 & 0.294 & 2508 & 0.954 & pass \\
 two-sided ($|\cdot|$) & 1e-05 & 0.850 & 1.827 & 1.653 & 1.959 & 0.306 & 2485 & 0.986 & pass \\
 two-sided ($|\cdot|$) & 1e-05 & 0.900 & 1.646 & 1.551 & 1.947 & 0.397 & 2467 & 0.942 & pass \\
 two-sided ($|\cdot|$) & 1e-05 & 0.950 & 1.748 & 1.594 & 1.869 & 0.274 & 2461 & 0.974 & pass \\
 two-sided ($|\cdot|$) & 1e-05 & 1.000 & 1.823 & 1.662 & 1.913 & 0.251 & 2453 & 0.990 & pass \\
 two-sided ($|\cdot|$) & 1e-05 & 1.050 & 1.813 & 1.646 & 1.928 & 0.282 & 2442 & 0.986 & pass \\
 two-sided ($|\cdot|$) & 1e-05 & 1.100 & 1.769 & 1.662 & 1.979 & 0.318 & 2441 & 0.975 & pass \\
 two-sided ($|\cdot|$) & 1e-05 & 1.150 & 1.700 & 1.632 & 2.017 & 0.385 & 2446 & 0.938 & pass \\
 two-sided ($|\cdot|$) & 1e-05 & 1.200 & 1.809 & 1.693 & 2.026 & 0.333 & 2414 & 0.958 & pass \\
 two-sided ($|\cdot|$) & 1e-05 & 1.300 & 1.622 & 1.445 & 1.878 & 0.433 & 2377 & 0.989 & pass \\
 two-sided ($|\cdot|$) & 2e-05 & 0.700 & 1.703 & 1.640 & 1.833 & 0.192 & 4653 & 0.926 & pass \\
 two-sided ($|\cdot|$) & 2e-05 & 0.800 & 1.760 & 1.705 & 1.858 & 0.153 & 4568 & 0.982 & pass \\
 two-sided ($|\cdot|$) & 2e-05 & 0.850 & 1.760 & 1.719 & 1.854 & 0.135 & 4512 & 0.984 & pass \\
 two-sided ($|\cdot|$) & 2e-05 & 0.900 & 1.761 & 1.707 & 1.859 & 0.152 & 4461 & 0.982 & pass \\
 two-sided ($|\cdot|$) & 2e-05 & 0.950 & 1.787 & 1.726 & 1.876 & 0.150 & 4455 & 0.978 & pass \\
 two-sided ($|\cdot|$) & 2e-05 & 1.000 & 1.714 & 1.666 & 1.840 & 0.174 & 4449 & 0.970 & pass \\
 two-sided ($|\cdot|$) & 2e-05 & 1.050 & 1.667 & 1.626 & 1.799 & 0.174 & 4414 & 0.969 & pass \\
 two-sided ($|\cdot|$) & 2e-05 & 1.100 & 1.701 & 1.659 & 1.791 & 0.132 & 4423 & 0.979 & pass \\
 two-sided ($|\cdot|$) & 2e-05 & 1.150 & 1.688 & 1.641 & 1.776 & 0.135 & 4405 & 0.976 & pass \\
 two-sided ($|\cdot|$) & 2e-05 & 1.200 & 1.655 & 1.623 & 1.776 & 0.153 & 4354 & 0.978 & pass \\
 two-sided ($|\cdot|$) & 2e-05 & 1.300 & 1.637 & 1.619 & 1.730 & 0.110 & 4328 & 0.984 & pass \\
 two-sided ($|\cdot|$) & 4e-05 & 0.700 & 1.646 & 1.605 & 1.717 & 0.112 & 4868 & 0.968 & pass \\
 two-sided ($|\cdot|$) & 4e-05 & 0.800 & 1.713 & 1.673 & 1.770 & 0.097 & 4791 & 0.975 & pass \\
 two-sided ($|\cdot|$) & 4e-05 & 0.850 & 1.665 & 1.650 & 1.736 & 0.086 & 4831 & 0.977 & pass \\
 two-sided ($|\cdot|$) & 4e-05 & 0.900 & 1.664 & 1.628 & 1.728 & 0.100 & 4827 & 0.980 & pass \\
 two-sided ($|\cdot|$) & 4e-05 & 0.950 & 1.642 & 1.626 & 1.712 & 0.085 & 4860 & 0.976 & pass \\
 two-sided ($|\cdot|$) & 4e-05 & 1.000 & 1.687 & 1.652 & 1.744 & 0.092 & 4846 & 0.976 & pass \\
 two-sided ($|\cdot|$) & 4e-05 & 1.050 & 1.660 & 1.636 & 1.721 & 0.085 & 4865 & 0.980 & pass \\
 two-sided ($|\cdot|$) & 4e-05 & 1.100 & 1.669 & 1.632 & 1.733 & 0.100 & 4797 & 0.977 & pass \\
 two-sided ($|\cdot|$) & 4e-05 & 1.150 & 1.655 & 1.620 & 1.712 & 0.093 & 4784 & 0.976 & pass \\
 two-sided ($|\cdot|$) & 4e-05 & 1.200 & 1.658 & 1.630 & 1.725 & 0.095 & 4848 & 0.970 & pass \\
 two-sided ($|\cdot|$) & 4e-05 & 1.300 & 1.622 & 1.600 & 1.688 & 0.087 & 4912 & 0.977 & pass \\
 two-sided ($|\cdot|$) & 8e-05 & 0.700 & 1.625 & 1.601 & 1.691 & 0.090 & 2537 & 0.953 & pass \\
 two-sided ($|\cdot|$) & 8e-05 & 0.800 & 1.625 & 1.604 & 1.687 & 0.082 & 2540 & 0.959 & pass \\
 two-sided ($|\cdot|$) & 8e-05 & 0.850 & 1.625 & 1.607 & 1.683 & 0.076 & 2543 & 0.957 & pass \\
 two-sided ($|\cdot|$) & 8e-05 & 0.900 & 1.605 & 1.583 & 1.666 & 0.083 & 2611 & 0.949 & pass \\
 two-sided ($|\cdot|$) & 8e-05 & 0.950 & 1.636 & 1.595 & 1.690 & 0.095 & 2609 & 0.958 & pass \\
 two-sided ($|\cdot|$) & 8e-05 & 1.000 & 1.639 & 1.607 & 1.696 & 0.089 & 2627 & 0.953 & pass \\
 two-sided ($|\cdot|$) & 8e-05 & 1.050 & 1.662 & 1.627 & 1.712 & 0.085 & 2706 & 0.961 & pass \\
 two-sided ($|\cdot|$) & 8e-05 & 1.100 & 1.654 & 1.617 & 1.703 & 0.086 & 2735 & 0.957 & pass \\
 two-sided ($|\cdot|$) & 8e-05 & 1.150 & 1.646 & 1.612 & 1.705 & 0.093 & 2751 & 0.958 & pass \\
 two-sided ($|\cdot|$) & 8e-05 & 1.200 & 1.676 & 1.638 & 1.716 & 0.078 & 2810 & 0.967 & pass \\
 two-sided ($|\cdot|$) & 8e-05 & 1.300 & 1.663 & 1.627 & 1.718 & 0.091 & 2838 & 0.961 & pass \\
\end{longtable}
\end{center}


\subsection{Replication summary (base vs instruct)}
\begin{table}[H]
\centering
\scriptsize
\setlength{\tabcolsep}{2pt}
\resizebox{\linewidth}{!}{%
\begin{tabular}{llrrrrrr}
\toprule
Spike def & Target rate & $g^{\star}_{\mathrm{base}}$ & $g^{\star}_{\mathrm{inst}}$ & $\Delta b_{\mathrm{tot}}$ (base) & $\Delta b_{\mathrm{tot}}$ (inst) & $\chi$ (base) & $\chi$ (inst) \\
\midrule
 one-sided (+) & 1e-05 & 0.700 & 0.700 & 0.274 & 0.266 & 154.531 & 99.469 \\
 one-sided (+) & 2e-05 & 0.700 & 0.700 & 0.276 & 0.271 & 251.222 & 160.923 \\
 one-sided (+) & 4e-05 & 1.300 & 1.300 & 0.217 & 0.210 & 361.316 & 242.203 \\
 one-sided (+) & 8e-05 & 1.300 & 1.300 & 0.148 & 0.145 & 562.081 & 338.812 \\
 two-sided ($|\cdot|$) & 1e-05 & 0.800 & 0.850 & 0.304 & 0.300 & 196.479 & 119.844 \\
 two-sided ($|\cdot|$) & 2e-05 & 0.800 & 1.000 & 0.313 & 0.305 & 316.137 & 201.021 \\
 two-sided ($|\cdot|$) & 4e-05 & 1.300 & 1.300 & 0.264 & 0.253 & 458.077 & 329.981 \\
 two-sided ($|\cdot|$) & 8e-05 & 1.300 & 1.300 & 0.188 & 0.182 & 726.533 & 491.691 \\
\bottomrule
\end{tabular}%
}
\caption{Replication summary comparing base vs instruction-tuned checkpoints at their respective $g^{\star}$ (appendix).}
\label{tab:replication}
\end{table}


\subsection{Selected Dataset A condition table}
% NOTE: This table is typeset as a longtable to avoid float-too-large warnings in the appendix.
\begin{center}
\tiny
\setlength{\tabcolsep}{1.5pt}
\renewcommand{\arraystretch}{1.05}
\setlength{\LTpre}{0pt}
\setlength{\LTpost}{0pt}
\begin{longtable}{llrrrrrrrrrr}
\caption{Selected columns from Table~T01 (Dataset A): rate-matching error, branching, null-controlled residual, susceptibility proxy, crackling exponent estimate, and component summary statistics across all conditions.}\label{tab:t01_selected}\\
\toprule
Spike def & Target rate & $g$ & max|rate err| & $b_{\mathrm{tot}}$ & $\Delta b_{\mathrm{tot}}$ & $\chi$ & $\gamma$ & \#avals & mean size & mean span (tok) & mean span (layers) \\
\midrule
\endfirsthead
\multicolumn{12}{l}{\textbf{Table \thetable} (continued)}\\
\toprule
Spike def & Target rate & $g$ & max|rate err| & $b_{\mathrm{tot}}$ & $\Delta b_{\mathrm{tot}}$ & $\chi$ & $\gamma$ & \#avals & mean size & mean span (tok) & mean span (layers) \\
\midrule
\endhead
\midrule
\multicolumn{12}{r}{\footnotesize Continued on next page}\\
\endfoot
\bottomrule
\endlastfoot
one-sided (+) & 1e-05 & 0.700 & 9.63e-08 & 0.599 & 0.266 & 99.469 & 1.745 & 30584 & 2.833 & 1.442 & 1.529 \\
 one-sided (+) & 1e-05 & 0.800 & 9.95e-08 & 0.597 & 0.263 & 98.227 & 1.718 & 30662 & 2.825 & 1.431 & 1.539 \\
 one-sided (+) & 1e-05 & 0.850 & 8.44e-08 & 0.594 & 0.261 & 97.997 & 1.814 & 30786 & 2.816 & 1.424 & 1.539 \\
 one-sided (+) & 1e-05 & 0.900 & 7.62e-08 & 0.592 & 0.259 & 100.388 & 1.793 & 30882 & 2.806 & 1.418 & 1.541 \\
 one-sided (+) & 1e-05 & 0.950 & 9.63e-08 & 0.594 & 0.262 & 100.642 & 1.792 & 30771 & 2.815 & 1.417 & 1.548 \\
 one-sided (+) & 1e-05 & 1.000 & 1.12e-07 & 0.593 & 0.263 & 103.465 & 1.686 & 30794 & 2.813 & 1.414 & 1.549 \\
 one-sided (+) & 1e-05 & 1.050 & 7.76e-08 & 0.592 & 0.260 & 104.412 & 1.682 & 30710 & 2.822 & 1.412 & 1.552 \\
 one-sided (+) & 1e-05 & 1.100 & 9.67e-08 & 0.589 & 0.255 & 103.114 & 1.632 & 30817 & 2.810 & 1.408 & 1.548 \\
 one-sided (+) & 1e-05 & 1.150 & 8.92e-08 & 0.588 & 0.257 & 101.886 & 1.702 & 30820 & 2.810 & 1.407 & 1.548 \\
 one-sided (+) & 1e-05 & 1.200 & 8.28e-08 & 0.585 & 0.252 & 101.261 & 1.730 & 30903 & 2.804 & 1.404 & 1.545 \\
 one-sided (+) & 1e-05 & 1.300 & 9.95e-08 & 0.580 & 0.246 & 103.526 & 1.702 & 31031 & 2.792 & 1.395 & 1.540 \\
 one-sided (+) & 2e-05 & 0.700 & 1.78e-07 & 0.833 & 0.271 & 160.923 & 1.726 & 33555 & 5.163 & 1.713 & 1.913 \\
 one-sided (+) & 2e-05 & 0.800 & 1.62e-07 & 0.830 & 0.270 & 163.233 & 1.729 & 33686 & 5.144 & 1.700 & 1.922 \\
 one-sided (+) & 2e-05 & 0.850 & 2.18e-07 & 0.829 & 0.267 & 162.730 & 1.751 & 33711 & 5.136 & 1.693 & 1.921 \\
 one-sided (+) & 2e-05 & 0.900 & 2.01e-07 & 0.828 & 0.266 & 163.570 & 1.708 & 33741 & 5.131 & 1.690 & 1.928 \\
 one-sided (+) & 2e-05 & 0.950 & 2.34e-07 & 0.824 & 0.263 & 163.832 & 1.598 & 33920 & 5.104 & 1.686 & 1.932 \\
 one-sided (+) & 2e-05 & 1.000 & 1.71e-07 & 0.823 & 0.260 & 165.146 & 1.613 & 33974 & 5.098 & 1.682 & 1.931 \\
 one-sided (+) & 2e-05 & 1.050 & 2.04e-07 & 0.822 & 0.259 & 164.680 & 1.670 & 33948 & 5.102 & 1.680 & 1.930 \\
 one-sided (+) & 2e-05 & 1.100 & 2.36e-07 & 0.820 & 0.257 & 162.738 & 1.686 & 34009 & 5.096 & 1.679 & 1.931 \\
 one-sided (+) & 2e-05 & 1.150 & 1.91e-07 & 0.818 & 0.258 & 158.867 & 1.729 & 34110 & 5.077 & 1.679 & 1.928 \\
 one-sided (+) & 2e-05 & 1.200 & 2.47e-07 & 0.814 & 0.252 & 159.493 & 1.565 & 34292 & 5.050 & 1.672 & 1.921 \\
 one-sided (+) & 2e-05 & 1.300 & 2.41e-07 & 0.810 & 0.251 & 165.867 & 1.685 & 34493 & 5.018 & 1.662 & 1.911 \\
 one-sided (+) & 4e-05 & 0.700 & 4.58e-07 & 1.109 & 0.233 & 255.178 & 1.652 & 25770 & 13.436 & 2.144 & 2.500 \\
 one-sided (+) & 4e-05 & 0.800 & 4.28e-07 & 1.108 & 0.233 & 255.887 & 1.684 & 25745 & 13.439 & 2.136 & 2.519 \\
 one-sided (+) & 4e-05 & 0.850 & 4.31e-07 & 1.108 & 0.230 & 254.088 & 1.675 & 25728 & 13.459 & 2.132 & 2.530 \\
 one-sided (+) & 4e-05 & 0.900 & 4.17e-07 & 1.106 & 0.232 & 253.071 & 1.683 & 25908 & 13.366 & 2.124 & 2.523 \\
 one-sided (+) & 4e-05 & 0.950 & 4.08e-07 & 1.104 & 0.230 & 252.461 & 1.639 & 26049 & 13.294 & 2.117 & 2.524 \\
 one-sided (+) & 4e-05 & 1.000 & 3.66e-07 & 1.102 & 0.226 & 256.167 & 1.669 & 26230 & 13.200 & 2.102 & 2.506 \\
 one-sided (+) & 4e-05 & 1.050 & 4.06e-07 & 1.100 & 0.224 & 257.204 & 1.650 & 26239 & 13.190 & 2.104 & 2.506 \\
 one-sided (+) & 4e-05 & 1.100 & 5.23e-07 & 1.098 & 0.223 & 252.135 & 1.602 & 26387 & 13.110 & 2.106 & 2.494 \\
 one-sided (+) & 4e-05 & 1.150 & 4.60e-07 & 1.095 & 0.219 & 249.300 & 1.624 & 26563 & 13.025 & 2.104 & 2.491 \\
 one-sided (+) & 4e-05 & 1.200 & 5.20e-07 & 1.093 & 0.218 & 244.250 & 1.645 & 26786 & 12.911 & 2.095 & 2.473 \\
 one-sided (+) & 4e-05 & 1.300 & 4.65e-07 & 1.089 & 0.210 & 242.203 & 1.647 & 27007 & 12.821 & 2.094 & 2.458 \\
 one-sided (+) & 8e-05 & 0.700 & 1.01e-06 & 1.402 & 0.163 & 397.668 & 1.685 & 12101 & 57.153 & 2.945 & 2.740 \\
 one-sided (+) & 8e-05 & 0.800 & 1.06e-06 & 1.399 & 0.164 & 395.740 & 1.736 & 12377 & 55.863 & 2.921 & 2.799 \\
 one-sided (+) & 8e-05 & 0.850 & 9.29e-07 & 1.399 & 0.164 & 393.898 & 1.703 & 12370 & 55.913 & 2.926 & 2.825 \\
 one-sided (+) & 8e-05 & 0.900 & 1.12e-06 & 1.398 & 0.162 & 392.063 & 1.725 & 12412 & 55.701 & 2.931 & 2.812 \\
 one-sided (+) & 8e-05 & 0.950 & 9.36e-07 & 1.397 & 0.160 & 383.967 & 1.705 & 12536 & 55.174 & 2.912 & 2.801 \\
 one-sided (+) & 8e-05 & 1.000 & 8.98e-07 & 1.395 & 0.158 & 381.054 & 1.717 & 12629 & 54.784 & 2.908 & 2.781 \\
 one-sided (+) & 8e-05 & 1.050 & 9.75e-07 & 1.395 & 0.156 & 374.196 & 1.712 & 12635 & 54.778 & 2.928 & 2.725 \\
 one-sided (+) & 8e-05 & 1.100 & 1.06e-06 & 1.392 & 0.154 & 366.487 & 1.709 & 12811 & 53.966 & 2.912 & 2.712 \\
 one-sided (+) & 8e-05 & 1.150 & 8.73e-07 & 1.392 & 0.152 & 353.791 & 1.698 & 12788 & 54.079 & 2.924 & 2.681 \\
 one-sided (+) & 8e-05 & 1.200 & 9.91e-07 & 1.390 & 0.149 & 344.150 & 1.708 & 12939 & 53.458 & 2.918 & 2.653 \\
 one-sided (+) & 8e-05 & 1.300 & 1.12e-06 & 1.387 & 0.145 & 338.812 & 1.727 & 13127 & 52.655 & 2.904 & 2.594 \\
 two-sided ($|\cdot|$) & 1e-05 & 0.700 & 7.60e-08 & 0.626 & 0.293 & 122.941 & 1.736 & 28576 & 3.035 & 1.426 & 1.595 \\
 two-sided ($|\cdot|$) & 1e-05 & 0.800 & 5.91e-08 & 0.627 & 0.297 & 121.347 & 1.719 & 28573 & 3.035 & 1.413 & 1.606 \\
 two-sided ($|\cdot|$) & 1e-05 & 0.850 & 4.30e-08 & 0.629 & 0.300 & 119.844 & 1.827 & 28483 & 3.046 & 1.412 & 1.616 \\
 two-sided ($|\cdot|$) & 1e-05 & 0.900 & 4.66e-08 & 0.629 & 0.298 & 120.649 & 1.646 & 28480 & 3.046 & 1.410 & 1.622 \\
 two-sided ($|\cdot|$) & 1e-05 & 0.950 & 4.70e-08 & 0.627 & 0.294 & 119.745 & 1.748 & 28530 & 3.040 & 1.408 & 1.626 \\
 two-sided ($|\cdot|$) & 1e-05 & 1.000 & 3.65e-08 & 0.627 & 0.295 & 120.883 & 1.823 & 28561 & 3.039 & 1.405 & 1.632 \\
 two-sided ($|\cdot|$) & 1e-05 & 1.050 & 4.66e-08 & 0.623 & 0.295 & 120.902 & 1.813 & 28622 & 3.031 & 1.400 & 1.632 \\
 two-sided ($|\cdot|$) & 1e-05 & 1.100 & 5.71e-08 & 0.621 & 0.295 & 119.962 & 1.769 & 28698 & 3.022 & 1.399 & 1.633 \\
 two-sided ($|\cdot|$) & 1e-05 & 1.150 & 4.73e-08 & 0.619 & 0.291 & 118.353 & 1.700 & 28748 & 3.018 & 1.397 & 1.633 \\
 two-sided ($|\cdot|$) & 1e-05 & 1.200 & 5.33e-08 & 0.618 & 0.287 & 120.059 & 1.809 & 28703 & 3.024 & 1.393 & 1.636 \\
 two-sided ($|\cdot|$) & 1e-05 & 1.300 & 6.62e-08 & 0.614 & 0.284 & 124.299 & 1.622 & 28686 & 3.024 & 1.388 & 1.630 \\
 two-sided ($|\cdot|$) & 2e-05 & 0.700 & 1.63e-07 & 0.851 & 0.306 & 197.402 & 1.703 & 31375 & 5.529 & 1.684 & 1.997 \\
 two-sided ($|\cdot|$) & 2e-05 & 0.800 & 1.23e-07 & 0.851 & 0.301 & 194.726 & 1.760 & 31338 & 5.536 & 1.673 & 2.013 \\
 two-sided ($|\cdot|$) & 2e-05 & 0.850 & 9.25e-08 & 0.852 & 0.304 & 193.456 & 1.760 & 31302 & 5.545 & 1.668 & 2.023 \\
 two-sided ($|\cdot|$) & 2e-05 & 0.900 & 9.08e-08 & 0.852 & 0.304 & 196.842 & 1.761 & 31306 & 5.541 & 1.664 & 2.024 \\
 two-sided ($|\cdot|$) & 2e-05 & 0.950 & 1.00e-07 & 0.853 & 0.302 & 196.673 & 1.787 & 31235 & 5.555 & 1.662 & 2.034 \\
 two-sided ($|\cdot|$) & 2e-05 & 1.000 & 6.21e-08 & 0.853 & 0.305 & 201.021 & 1.714 & 31188 & 5.566 & 1.660 & 2.036 \\
 two-sided ($|\cdot|$) & 2e-05 & 1.050 & 9.85e-08 & 0.852 & 0.303 & 202.030 & 1.667 & 31240 & 5.555 & 1.655 & 2.031 \\
 two-sided ($|\cdot|$) & 2e-05 & 1.100 & 7.44e-08 & 0.850 & 0.299 & 200.626 & 1.701 & 31241 & 5.556 & 1.654 & 2.039 \\
 two-sided ($|\cdot|$) & 2e-05 & 1.150 & 1.21e-07 & 0.848 & 0.299 & 197.394 & 1.688 & 31283 & 5.547 & 1.652 & 2.038 \\
 two-sided ($|\cdot|$) & 2e-05 & 1.200 & 1.23e-07 & 0.844 & 0.298 & 198.463 & 1.655 & 31381 & 5.529 & 1.647 & 2.034 \\
 two-sided ($|\cdot|$) & 2e-05 & 1.300 & 1.07e-07 & 0.840 & 0.289 & 204.731 & 1.637 & 31581 & 5.494 & 1.639 & 2.020 \\
 two-sided ($|\cdot|$) & 4e-05 & 0.700 & 2.33e-07 & 1.118 & 0.272 & 324.517 & 1.646 & 24599 & 14.106 & 2.094 & 2.637 \\
 two-sided ($|\cdot|$) & 4e-05 & 0.800 & 2.29e-07 & 1.118 & 0.271 & 325.905 & 1.713 & 24603 & 14.094 & 2.078 & 2.646 \\
 two-sided ($|\cdot|$) & 4e-05 & 0.850 & 2.49e-07 & 1.118 & 0.271 & 323.725 & 1.665 & 24604 & 14.096 & 2.077 & 2.657 \\
 two-sided ($|\cdot|$) & 4e-05 & 0.900 & 1.94e-07 & 1.117 & 0.270 & 326.791 & 1.664 & 24661 & 14.071 & 2.072 & 2.647 \\
 two-sided ($|\cdot|$) & 4e-05 & 0.950 & 1.89e-07 & 1.114 & 0.266 & 325.458 & 1.642 & 24815 & 13.984 & 2.065 & 2.642 \\
 two-sided ($|\cdot|$) & 4e-05 & 1.000 & 2.05e-07 & 1.113 & 0.267 & 330.934 & 1.687 & 24957 & 13.902 & 2.057 & 2.637 \\
 two-sided ($|\cdot|$) & 4e-05 & 1.050 & 1.94e-07 & 1.113 & 0.267 & 329.652 & 1.660 & 24983 & 13.880 & 2.060 & 2.624 \\
 two-sided ($|\cdot|$) & 4e-05 & 1.100 & 2.12e-07 & 1.112 & 0.263 & 324.618 & 1.669 & 24968 & 13.899 & 2.063 & 2.622 \\
 two-sided ($|\cdot|$) & 4e-05 & 1.150 & 2.10e-07 & 1.111 & 0.263 & 318.179 & 1.655 & 24982 & 13.883 & 2.063 & 2.619 \\
 two-sided ($|\cdot|$) & 4e-05 & 1.200 & 2.07e-07 & 1.107 & 0.258 & 319.883 & 1.658 & 25226 & 13.758 & 2.058 & 2.607 \\
 two-sided ($|\cdot|$) & 4e-05 & 1.300 & 2.54e-07 & 1.103 & 0.253 & 329.981 & 1.622 & 25543 & 13.583 & 2.055 & 2.583 \\
 two-sided ($|\cdot|$) & 8e-05 & 0.700 & 4.33e-07 & 1.388 & 0.199 & 522.920 & 1.625 & 12658 & 54.793 & 2.815 & 3.152 \\
 two-sided ($|\cdot|$) & 8e-05 & 0.800 & 4.64e-07 & 1.389 & 0.199 & 516.944 & 1.625 & 12695 & 54.648 & 2.795 & 3.187 \\
 two-sided ($|\cdot|$) & 8e-05 & 0.850 & 5.76e-07 & 1.389 & 0.199 & 511.292 & 1.625 & 12756 & 54.394 & 2.788 & 3.188 \\
 two-sided ($|\cdot|$) & 8e-05 & 0.900 & 4.55e-07 & 1.388 & 0.201 & 508.814 & 1.605 & 12846 & 53.996 & 2.793 & 3.199 \\
 two-sided ($|\cdot|$) & 8e-05 & 0.950 & 4.49e-07 & 1.387 & 0.199 & 510.891 & 1.636 & 12878 & 53.891 & 2.787 & 3.202 \\
 two-sided ($|\cdot|$) & 8e-05 & 1.000 & 4.44e-07 & 1.386 & 0.197 & 516.272 & 1.639 & 13016 & 53.331 & 2.778 & 3.150 \\
 two-sided ($|\cdot|$) & 8e-05 & 1.050 & 5.57e-07 & 1.384 & 0.196 & 517.214 & 1.662 & 13115 & 52.903 & 2.786 & 3.133 \\
 two-sided ($|\cdot|$) & 8e-05 & 1.100 & 5.28e-07 & 1.382 & 0.191 & 509.224 & 1.654 & 13248 & 52.353 & 2.788 & 3.104 \\
 two-sided ($|\cdot|$) & 8e-05 & 1.150 & 5.94e-07 & 1.382 & 0.190 & 502.037 & 1.646 & 13313 & 52.082 & 2.791 & 3.058 \\
 two-sided ($|\cdot|$) & 8e-05 & 1.200 & 4.34e-07 & 1.381 & 0.188 & 495.254 & 1.676 & 13496 & 51.433 & 2.790 & 3.047 \\
 two-sided ($|\cdot|$) & 8e-05 & 1.300 & 6.07e-07 & 1.378 & 0.182 & 491.691 & 1.663 & 13581 & 51.068 & 2.815 & 2.931 \\
\end{longtable}
\end{center}


\subsection{Artifact provenance}
\begin{table*}[t]
\centering
\footnotesize
\setlength{\tabcolsep}{3pt}
\resizebox{\textwidth}{!}{%
\begin{tabular}{lll}
\toprule
Artifact & Run & Config hash (prefix)\\
\midrule
 \texttt{F01\_RASTER\_EXAMPLE} & \texttt{RUN\_S05\_ae9d76afcc57} & \texttt{d7513f37dfef...} \\
 \texttt{F02\_RATE\_MATCH\_CHECK} & \texttt{RUN\_S05\_ae9d76afcc57} & \texttt{d7513f37dfef...} \\
 \texttt{F03\_BRANCHING\_CURVES} & \texttt{RUN\_S05\_ae9d76afcc57} & \texttt{d7513f37dfef...} \\
 \texttt{F04\_NULL\_DELTAB} & \texttt{RUN\_S05\_ae9d76afcc57} & \texttt{d7513f37dfef...} \\
 \texttt{F05\_GSTAR\_SELECTION} & \texttt{RUN\_S05\_ae9d76afcc57} & \texttt{d7513f37dfef...} \\
 \texttt{F06\_GENERALIZATION\_B} & \texttt{RUN\_S06B\_1791d65dc967} & \texttt{d7513f37dfef...} \\
 \texttt{F07\_ARC\_MCQ} & \texttt{RUN\_S06ARC\_6fd0209c7daf} & \texttt{d7513f37dfef...} \\
 \texttt{F08\_SPIKEDEF\_ROBUST} & \texttt{RUN\_S05\_ae9d76afcc57} & \texttt{d7513f37dfef...} \\
 \texttt{F09\_CHI\_CURVES} & \texttt{RUN\_S05\_ae9d76afcc57} & \texttt{d7513f37dfef...} \\
 \texttt{F10\_NULL\_COMPARE} & \texttt{RUN\_S05\_ae9d76afcc57} & \texttt{d7513f37dfef...} \\
 \texttt{F11\_ABLATIONS} & \texttt{RUN\_S05\_ae9d76afcc57} & \texttt{d7513f37dfef...} \\
 \texttt{T01\_SUMMARY} & \texttt{RUN\_S05\_ae9d76afcc57} & \texttt{d7513f37dfef...} \\
 \texttt{T02\_GENERALIZATION} & \texttt{RUN\_S06B\_1791d65dc967} & \texttt{d7513f37dfef...} \\
 \texttt{T03\_ARC} & \texttt{RUN\_S06ARC\_6fd0209c7daf} & \texttt{d7513f37dfef...} \\
 \texttt{T04\_TAIL\_FITS} & \texttt{RUN\_S05\_ae9d76afcc57} & \texttt{d7513f37dfef...} \\
 \texttt{T05\_CRACKLING\_DIAGNOSTICS} & \texttt{RUN\_S05\_ae9d76afcc57} & \texttt{d7513f37dfef...} \\
 \texttt{T06\_ABLATIONS} & \texttt{RUN\_S05\_ae9d76afcc57} & \texttt{d7513f37dfef...} \\
 \texttt{T07\_REPLICATION\_SUMMARY} & \texttt{RUN\_S07\_72cab7151a1f} & \texttt{d7513f37dfef...} \\
\bottomrule
\end{tabular}%
}
\caption{Artifact provenance (run identifiers and resolved config hashes) for exported figures and tables. Full hashes and checksums are recorded in each run's \texttt{run\_record.json} and in \texttt{MANIFEST.sha256}.}
\label{tab:provenance}
\end{table*}


\end{document}
